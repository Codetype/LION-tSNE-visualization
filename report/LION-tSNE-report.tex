\documentclass{article}


\usepackage{arxiv}

\usepackage[utf8]{inputenc} % allow utf-8 input
\usepackage[T1]{fontenc}    % use 8-bit T1 fonts
\usepackage{hyperref}       % hyperlinks
\usepackage{url}            % simple URL typesetting
\usepackage{booktabs}       % professional-quality tables
\usepackage{amsfonts}       % blackboard math symbols
\usepackage{nicefrac}       % compact symbols for 1/2, etc.
\usepackage{microtype}      % microtypography
\usepackage{lipsum}
\usepackage{graphicx}
\graphicspath{ {./images/} }


\title{k-NN Sampling for Visualization of Dynamic Data Using LION-tSNE - analysis}


\author{
 Gędłek Paweł \\
  Wydział Informatyki, Elektroniki i Telekomunikacji\\
  Akademia Górniczo-Hutnicza \\
  Kraków \\
  \texttt{gedlek@student.agh.edu.pl} \\
  \And
Wójtowicz Patryk \\
  Wydział Informatyki, Elektroniki i Telekomunikacji\\
  Akademia Górniczo-Hutnicza \\
  Kraków \\
  \texttt{wojtowicz@student.agh.edu.pl} \\
}


\begin{document}
\maketitle
\begin{abstract}
TODO
\end{abstract}


% keywords can be removed
% keywords{First keyword \and Second keyword \and More}


\section{Struktura raportu}
\label{sec:report_structure}
\paragraph{}
\tableofcontents

\section{Metoda tSNE}
\label{sec:tSNE}
\paragraph{}
\subsection{Czym właściwie jest tSNE?}
Algorytm tSNE(t-Distributed Stochastic Neighbor Embedding) którego
autorami sa Laurens van der Maaten oraz Geoffrey Hinton bazuje na
metodzie SNE, której głównym załozeniem jest reprezentacja
wielowymiarowych danych w mozliwy do zobrazowania dla człowieka dwulub
trzy-wymiarowej przestrzeni. Osiaga sie to poprzez modelowanie wysoko
wymiarowych obiektów poprzez dwu- lub trzy-wymiarowe punkty w taki
sposób, ze zblizone obiekty modelowane sa poprzez bliskie sobie punkty, a
oddalone obiekty modelowane są poprzez oddalone od siebie punkty z
duzym prawdopodobienstwem.

\subsection{Algorytm tSNE - podstawy matematyczne}
\begin{itemize}
\item 
Algorytm tSNE konwertuje odległosci miedzy parami punktów w
funkcje rozkładu prawdopodobienstwa okreslajaca podobienstwo
pomiedzy parami punktów.
\item 
Rozbieznosc miedzy podobienstwem wysoko wymiarowych danych z
nisko-wymiarowymi danymi jest mierzona poprzez dywergencje
Kullbacka-Leiblera i minimalizowana metoda gradientowa
poszukiwania minimum lokalnego
\end{itemize}

Mamy dany zbiór wejsciowy $X = {x1, x2...xn}$ gdzie dla kazdego $x_i \in R^{D}$
jest D-wymiarowym wektorem. Zbiór ten zostanie przekształcony do
postaci $Y = {y1, y2...yn}$ gdzie kazde $y_i \in R^d$ jest d-wymiarowym
wektorem oraz $d << D$ (zazwyczaj d = 2 lub 3). Podobienstwo pomiedzy
para punktów wejsciowych xi oraz xj oznaczamy poprzez pj/i , które jest
prawdopodobienstwem wybrania $x_j$ jako sasiada $x_i$ według funkcji gestosci
prawdopodobienstwa na rozkładzie normalnym gdzie $x_i$ stanowi centrum.
$p_{j/i}$ definiujemy jako:

...

\section{Metoda LION tSNE}
\label{sec:lionTSNE}
\paragraph{}

\section{kNN sampling}
\label{sec:kNN}
\paragraph{}

\section{Wnioski}
\label{sec:conclusions}
TODO

\bibliographystyle{unsrt}  
%\bibliography{references}  %%% Remove comment to use the external .bib file (using bibtex).
%%% and comment out the ``thebibliography'' section.


%%% Comment out this section when you \bibliography{references} is enabled.
\clearpage
\renewcommand\refname{Źródła}
\begin{thebibliography}{1}

\bibitem{paper} 
Bheekya Dharamsotu ; K. Swarupa Rani ; Salman Abdul Moiz ; C. Raghavendra Rao
\newblock Paper: k-NN Sampling for Visualization of Dynamic Data Using LION-tSNE. 
\newblock {\em https://ieeexplore.ieee.org/abstract/document/8990391}

\end{thebibliography}



\end{document}